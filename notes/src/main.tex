%%%%%%%%%%%%%%%%%%%%%%%%%%%%%%%%%%%%%%%%%
% Important note:
% Chapter heading images should have a 2:1 width:height ratio,
% e.g. 920px width and 460px height.
%
% The original template (the Legrand Orange Book Template) can be found here --> http://www.latextemplates.com/template/the-legrand-orange-book
%
% Original author of the Legrand Orange Book Template:
% Mathias Legrand (legrand.mathias@gmail.com) with modifications by:
% Vel (vel@latextemplates.com)
%
% Original License:
% CC BY-NC-SA 3.0 (http://creativecommons.org/licenses/by-nc-sa/3.0/)
%%%%%%%%%%%%%%%%%%%%%%%%%%%%%%%%%%%%%%%%%
 
%----------------------------------------------------------------------------------------
%	PACKAGES AND OTHER DOCUMENT CONFIGURATIONS
%----------------------------------------------------------------------------------------

\documentclass[11pt,fleqn]{book} % Default font size and left-justified equations

\usepackage[top=3cm,bottom=3cm,left=3.2cm,right=3.2cm,headsep=10pt,letterpaper]{geometry} % Page margins

\usepackage{xcolor} % Required for specifying colors by name
\definecolor{ocre}{RGB}{52,177,201} % Define the orange color used for highlighting throughout the book

% Font Settings
\usepackage{avant} % Use the Avantgarde font for headings
%\usepackage{times} % Use the Times font for headings
\usepackage{mathptmx} % Use the Adobe Times Roman as the default text font together with math symbols from the Sym­bol, Chancery and Com­puter Modern fonts

\usepackage{microtype} % Slightly tweak font spacing for aesthetics
\usepackage[utf8]{inputenc} % Required for including letters with accents
\usepackage[T1]{fontenc} % Use 8-bit encoding that has 256 glyphs

% Bibliography
\usepackage[style=alphabetic,sorting=nyt,sortcites=true,autopunct=true,babel=hyphen,hyperref=true,abbreviate=false,backref=true,backend=biber]{biblatex}
\addbibresource{bibliography.bib} % BibTeX bibliography file
\defbibheading{bibempty}{}

\input{structure} % Insert the commands.tex file which contains the majority of the structure behind the template

\begin{document}
\title{Clustering the interstellar medium}

%----------------------------------------------------------------------------------------
%	TITLE PAGE
%----------------------------------------------------------------------------------------

\begingroup
\thispagestyle{empty}
\AddToShipoutPicture*{\put(0,0){\includegraphics[scale=1.25]{esahubble}}} % Image background
\centering
\vspace*{5cm}
\par\normalfont\fontsize{35}{35}\sffamily\selectfont
\textbf{Calculus III}\\
{\LARGE github.com/mews6}\par % Book title
\vspace*{1cm}
{Jaime Torres}\par % Author name
\endgroup

%----------------------------------------------------------------------------------------
%	COPYRIGHT PAGE
%----------------------------------------------------------------------------------------

\newpage
~\vfill
\thispagestyle{empty}

%\noindent Copyright \copyright\ 2014 Andrea Hidalgo\\ % Copyright notice

\noindent \textit{First release, 2024} % Printing/edition date

%----------------------------------------------------------------------------------------
%	TABLE OF CONTENTS
%----------------------------------------------------------------------------------------

\chapterimage{head1.png} % Table of contents heading image

\pagestyle{empty} % No headers

\tableofcontents % Print the table of contents itself

%\cleardoublepage % Forces the first chapter to start on an odd page so it's on the right

\pagestyle{fancy} % Print headers again

%----------------------------------------------------------------------------------------
%	CHAPTER 1
%----------------------------------------------------------------------------------------

\chapterimage{head2.png} % Chapter heading image

\chapter{Introduction}

This is the persons i took the course with, they speak mostly spanish but they might help you!
\begin{itemize}
    \item Andres Angel (ja.angel908@uniandes.edu.co)    
\end{itemize}

\chapter{Linear Algebra Fundamentals}

In order to understand the concepts present in this module of calculus, a few preliminary concepts
in the realm of Linear Algebra are necessary, in order to not leave anybody lost, we'll be reviewing those topics.

\section{Lines}

A line is simply a mathematical object of the form $ y = mx + b $ that unites two points in the space we're using, through a straight path, where and b is the cutting point in x=0.
and m is $\tan{\alpha}$, or the slope of this line. We can imagine it as:

\begin{gather}
    m = \frac{y_1 - y_0}{x_1 - x_0} \\
\end{gather}

another equation of this object is, when working on $\mathbb{R}^2$
is defined as:

\begin{gather}
    ax + by = c
\end{gather}

The reason this sort of expression is so useful, is because this \textbf{is a general form.} In other words, we can define with this system every single possible line in $\mathbb{R^2}$.
For example, let's imagine a line that is perfectly vertical. Such as it can be explained, intuitively, as $x = 1$:

%Image 1

This vector can't be expressed through formula 1,1 or y = mx + b, because it wouldn't cross the '0' axis, and it would have an infinite slope. But if we arrange it through (1,2), we can say:

\begin{gather}
ax + by = c\\
by = c - ax\\
y = \frac{c}{b} + \frac{a}{b} x
\end{gather}


We can suppouse $\frac{c}{b}$ as the cutting point with y, and then imagine $\frac{a}{b}$ as 'm', we can then, use an example where we set 'b' to be 0
and define a situation where x = 1.

\subsection{Lines in $\mathbb{R}^n$}

We can define lines in more dimensions than $\mathbb{R}^2$
through different forms. For example, we can imagine it as a parametric equation of the form:

\begin{gather}
    p = t\vec{v}
\end{gather}

\paragraph*{Example 2.1.1}

Find the parametric equation that passes through p = (1,2,3) and is parallel to the vector (1,0,-1)

\textit{\textbf{Solution}}
\begin{gather}
    \begin{pmatrix}
        1 \\ 2 \\ 3
    \end{pmatrix} + t \begin{pmatrix}
        1 \\0 \\-1
    \end{pmatrix}
\end{gather}

With this form, we can imagine that the solution actually is producing 3 different equations, and every single one defines how this object will behave in a different dimension.

therefore:

\begin{gather}
    x(t) = 1 + t \\ 
    y(t) = 2 \\
    z(t) = 3-t
\end{gather}

Pretty neat, huh?

We can define such an equation in 'n' dimensions that can define a line going through be it a plane or a hyperplane like this. We only need two vectors of the same 'n' dimension.
So, taking this form, can we express it in other ways? Well, yeah! ...and we just did. 
the equation we generated


From here, we can also imagine that instead of using such a form, we can also equate everything to 't' and from here, we'll start defining a \textbf{Symetrical form of the line.}

In the previous example, we can imagine:
\begin{gather}
    x - 1 = y - 2 = \frac{z-3}{-1} = t    
\end{gather}

\section{Planes}

The plane can be defined as:

\begin{gather}
    ax + by + cz = d
\end{gather}

This plane can be defined with a point and a vector, for example:

\paragraph*{Example 2.2.1}
given $\vec{PQ} = (x,y,z)$ and $\vec{R} = (a,b,c)$, define a plane.

\textit{\textbf{Solution}}

\begin{gather}
    \begin{pmatrix}
        x - x_0 \\ y = y_0 \\ z - z_0
    \end{pmatrix} \cdot \begin{pmatrix}
        a\\b\\c
    \end{pmatrix} = 0\\
    a(x - x_0) + b(y - y_0) + c(z - z_0) = 0 \\
    ax - ax_0 + by - by_0 + cz - cz_0 = 0 \\
    ax + by + cz = ax_0 + by_0 + cz_0\\
    <d = ax_0 + by_0 + cz_0> \\
    ax + by + cz = d
\end{gather}

And from there we can define basically every plane in $\mathbb{R}^3$, as you can see, we arrived to the way we defined a plane in this dimension, and we could in theory, decide in a random point, a random vector, and start working from there into defining a plane.
But now, can we define a parametric equation of a plane, as we did with a line?

Well, the answer is yes.

the parametric form of a plane is actually pretty simple, as it is supremely similar to 
the one that defines a line, and is defined as:
\begin{gather}
    p + s\vec{v_1} + t \vec{v_2}
\end{gather}

if we generalize this into a system where R = (1,2,3), $v_1$ = (1,0,1) and $v_2$ = (1,1,0)
we can do a little example, written as:

\begin{gather}
    \begin{pmatrix}
        1\\2\\3
    \end{pmatrix} + s \begin{pmatrix}
        1\\0\\-1
    \end{pmatrix} + t\begin{pmatrix}
        1\\1\\0
    \end{pmatrix}
\end{gather}

\section{Vectors}


\end{document}